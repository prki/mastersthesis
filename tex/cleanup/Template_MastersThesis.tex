% arara: xelatex
% arara: xelatex
% arara: xelatex


% options:
% thesis=B bachelor's thesis
% thesis=M master's thesis
% czech thesis in Czech language
% english thesis in English language
% hidelinks remove colour boxes around hyperlinks

\documentclass[thesis=M,english]{FITthesis}[2012/10/20]

% \usepackage[utf8]{inputenc} % LaTeX source encoded as UTF-8
% \usepackage[latin2]{inputenc} % LaTeX source encoded as ISO-8859-2
% \usepackage[cp1250]{inputenc} % LaTeX source encoded as Windows-1250

\usepackage{graphicx} %graphics files inclusion
% \usepackage{subfig} %subfigures
% \usepackage{amsmath} %advanced maths
% \usepackage{amssymb} %additional math symbols

\usepackage{dirtree} %directory tree visualisation

% % list of acronyms
% \usepackage[acronym,nonumberlist,toc,numberedsection=autolabel]{glossaries}
% \iflanguage{czech}{\renewcommand*{\acronymname}{Seznam pou{\v z}it{\' y}ch zkratek}}{}
% \makeglossaries

% % % % % % % % % % % % % % % % % % % % % % % % % % % % % % 
% EDIT THIS
% % % % % % % % % % % % % % % % % % % % % % % % % % % % % % 

\department{Department of Theoretical Computer Science}
\title{Evaluating performance of an image compression scheme based on non-negative matrix factorization}
\authorGN{Marek} %author's given name/names
\authorFN{Pikna} %author's surname
\author{Marek Pikna} %author's name without academic degrees
\authorWithDegrees{Bc. Marek Pikna} %author's name with academic degrees
\supervisor{doc. Ing. Ivan Šimeček, Ph.D.}
\acknowledgements{THANKS (remove entirely in case you do not with to thank anyone)}
\abstractEN{Summarize the contents and contribution of your work in a few sentences in English language.}
\abstractCS{V n{\v e}kolika v{\v e}t{\' a}ch shr{\v n}te obsah a p{\v r}{\' i}nos t{\' e}to pr{\' a}ce v {\v c}esk{\' e}m jazyce.}
\placeForDeclarationOfAuthenticity{Prague}
\keywordsCS{Replace with comma-separated list of keywords in Czech.}
\keywordsEN{Replace with comma-separated list of keywords in English.}
\declarationOfAuthenticityOption{1} %select as appropriate, according to the desired license (integer 1-6)
% \website{http://site.example/thesis} %optional thesis URL

% dobry zdroje kompilace
% https://homepages.cae.wisc.edu/~ece533/project/f06/aguilera_rpt.pdf
% https://pdfs.semanticscholar.org/b481/96b92a6d28078b77f74f9129c3cd138ec291.pdf
% https://medium.com/@alfredayibonte/svd-and-image-compression-439c70745c99
% http://www.math.utah.edu/~goller/F15_M2270/BradyMathews_SVDImage.pdf

% TODO
% Introduction - nejspise prepsat budouci cas na pritomny (tato cast je popsana, nikoliv bude popsana etc)
% Na konci introduction kde se popipisuji kapitoly asi prolinkovat jednotoliva cisla, jakmile budou.
% Prepsat v uvodu JPEG z algoritmu na format (+ png)
% Doplnit bibliografii v introduction
% TODO thesis na NMF - https://www.researchgate.net/profile/Ngoc_Diep_Ho/publication/262258846_Nonnegative_matrix_factorization_algorithms_and_applications/links/02e7e537226cb7e59b000000.pdf - excelentni stuff	
% TODO potencialne popsat, proc je potreba nezapornost (viz uvod vyse)
% TODO kapitola, ktera vysvetluje zaklady maticovyho poctu, co je to nonnegative matrix, matrix
%          multiplication a tak? kam s touhle kapitolou? podsekce v NMF sekci nebo vlastni kapitola?
%           zatim bych to napsal jako vlastni sekci, kdyztak se to presune
% TODO zkontrolovat korektnost definice (velikosti), vysvetlit, co je to to r (rank - parametr nmf)
% TODO https://dial.uclouvain.be/pr/boreal/object/boreal:157009/datastream/PDF_01/view - zajimava vec o exaktnim NMF,
%      potencialne by byl gamechanger

\begin{document}

% \newacronym{CVUT}{{\v C}VUT}{{\v C}esk{\' e} vysok{\' e} u{\v c}en{\' i} technick{\' e} v Praze}
% \newacronym{FIT}{FIT}{Fakulta informa{\v c}n{\' i}ch technologi{\' i}}

\setsecnumdepth{part}
\chapter{Introduction}
Data encoding and consequently data compression are both problems which lie
at the heart of many modern technologies - digital television, videogames, mobile
communications, security cameras and all other kinds of multimedia. As the amount
of data only grows in the current world \emph{[CITATION NEEDED]}, the quality of
compression ends up becoming a very serious problem, since good compression can
very significantly reduce the costs of data storage as well as the costs and speed
of data transfer.
\\

To put the problem into perspective, in order to store images in the resolutions
currently considered as high resolution (1920x1080 pixels, the second most common
resolution on desktop devices \cite{res-stats}.
When storing an image with these dimensions using an uncompressed standard encoding,
the filesize would be almost 6 megabytes. Such a high size impacts many areas, such as
speed of transfer or storage costs. Fortunately, modern image compression formats such as
\emph{PNG} or \emph{JPG} are able to reduce this size remarkably.
\\

Currently, images contribute to the amount of data on the internet significantly - not only are images a common
form of media for professional purposes but some of the currently largest platofrms on the
internet are based on image sharing and image hosting. One modern social media platform centered around image
sharing had over 67 million new posts each day \cite{data-amount}.
Being able to obtain these images fast and in good quality is therefore highly important for both users, just
as it is important for the owners of these services to be able to store this data.
\\

In the recent years with growht related to machine learning and similar areas, such as artifical
intelligence, new applications of mathematical concepts were discovered. Some of these concepts
which are enjoying high success are algorithms related to \emph{dimensionality reduction}. These
algorithms aim to reduce the number of random variables under consideration \cite{dimens-reduction}.
\\

While these algorithms have been enjoying success mostly in areas such as data mining or machine
learning, their nature of reducing the amount of random variables under inspection means that the
algorithms are essentially also compression algorithms. One of the algorithms used for dimensionality
reduction and the one which this thesis focuses on is \emph{non-negative matrix factorization} (abbreviated
as \emph{NMF}). The \emph{NMF} algorithm is currently used in areas such as facial recognition or
astronomy. Heart of the algorithm is the factorization of a matrix consisting of non-negative values into
matrices.
\\

The research in image compression methods using dimensionality reduction algorithms is currently being
performed - another dimensionality reduction algorithm and its potential usage for image compression 
which has been well researched is the \emph{singular value decomposition} algorithm, for example in \cite{svd-compression}. This algorithm, just like the \emph{NMF}, factorizes a matrix - however, without
restricting the values to be non-negative. While certain similar research to see whether
\emph{NMF} can be used for image compression exists, the works are related to very
specific use-case scenarios.
\\

Thus, this thesis aims to analyze the potential of the \emph{NMF} algorithm
as a tool for image compression. In order to achieve this, the following points
will be explored in the thesis:
\begin{itemize}
  \item \emph{NMF} and its current applications will be studied (Chapter~\ref{ch:NMF}).
  \item The theory and practice related to digital image encoding and image
        compression will be explored and described together with
        modern image compression . (Chapters~\ref{ch:image-encoding} and~\ref{ch:image-compression})
\end{itemize}

By analyzing these concepts, a proof of concept image compression algorithm using
\emph{NMF} will be designed and implemented. By doing so, these issues will be
addressed:
\begin{itemize}
  \item Whether a certain way of representing an uncompressed image is better
        suited for non-negative matrix factorization.
  \item Utilizing both subjective as well as objective metrics commonly used
        for evaluating quality of image compression, the performance of
        the proof of concept compression scheme will be evaluated.
  \item How well suited \emph{NMF} is for usage as an algorithm for image
        compression.
\end{itemize} 
At the end of the thesis, the proof of concept algorithm will be compared to
the state of the art image compression algorithms and possible points related
to further analysis or will be explored.
\\

The first three chapters are related to the theoretical part of the problem, studying
\emph{NMF}, image encoding and image compression. The following chapters are related
to the practical part of this thesis - design and implementation of the image
compression algorithm and its evaluation.

\setsecnumdepth{all}
\chapter{Non-negative matrix factorization}
\label{ch:NMF}
This chapter discusses the \emph{non-negative matrix factorization} - defines
the problem, describes some of the existing solutions to the problem and
offers some observations. By doing so, the basics for the rest of the thesis are
provided.
\\
Non-negative matrix factorization as a problem was first formulated by Paatero
and Tapper in \emph{[CITATION NEEDED]}, but the works which have given
this problem far more popularity are the works of Lee and Seung \emph{[CITATION NEEDED]}, where
\emph{NMF} was applied to areas of machine learning and artificial
intelligence - more specifically to facial recognition and discovering semantic
features in encyclopedic articles.

\subsection{Problem definition}
Given a non-negative matrix \emph{V} with size $n \times m$, find non-negative matrix
 factors {W} with size $m \times r$ and {H} with size $r \times n$ such that:
\begin{equation}
  V \approx WH
\end{equation}

It should be noted here that the name of the problem might be misleading, as
the term \emph{"factorization"} is usually understood more as an exact
decomposition, whereas \emph{NMF} is in reality an approximation. Thus,
the problem is called \emph{non-negative matrix approximation} in certain
other works. \emph{[CITATION NEEDED  Inderjit S. Dhillon; Suvrit Sra (2005). Generalized Nonnegative Matrix Approximations with Bregman Divergences (PDF). NIPS.]}

\chapter{Digital image encoding}
\label{ch:image-encoding}
fgsfds

\chapter{Image compression}
\label{ch:image-compression}
fgsfds

\chapter{Realisation}

\setsecnumdepth{part}
\chapter{Conclusion}


\bibliographystyle{iso690}
\bibliography{mybibliographyfile}

\setsecnumdepth{all}
\appendix

\chapter{Acronyms}
% \printglossaries
\begin{description}
  \item[NMF] Non-negative matrix factorization
  \item[PNG] Portable Network Graphics
  \item[JPEG]
\end{description}


\chapter{Contents of enclosed CD}

%change appropriately

\begin{figure}
	\dirtree{%
		.1 readme.txt\DTcomment{the file with CD contents description}.
		.1 exe\DTcomment{the directory with executables}.
		.1 src\DTcomment{the directory of source codes}.
		.2 wbdcm\DTcomment{implementation sources}.
		.2 thesis\DTcomment{the directory of \LaTeX{} source codes of the thesis}.
		.1 text\DTcomment{the thesis text directory}.
		.2 thesis.pdf\DTcomment{the thesis text in PDF format}.
		.2 thesis.ps\DTcomment{the thesis text in PS format}.
	}
\end{figure}

\end{document}
